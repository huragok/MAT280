%%%%%%%%%%%%%%%%%%%%%%%%%%%%%%%%%%%%%%%%%
% Structured General Purpose Assignment
% LaTeX Template
%
% This template has been downloaded from:
% http://www.latextemplates.com
%
% Original author:
% Ted Pavlic (http://www.tedpavlic.com)
%
% Note:
% The \lipsum[#] commands throughout this template generate dummy text
% to fill the template out. These commands should all be removed when 
% writing assignment content.
%
%%%%%%%%%%%%%%%%%%%%%%%%%%%%%%%%%%%%%%%%%

%----------------------------------------------------------------------------------------
% PACKAGES AND OTHER DOCUMENT CONFIGURATIONS
%----------------------------------------------------------------------------------------

\documentclass{article}

\usepackage{fancyhdr} % Required for custom headers
\usepackage{lastpage} % Required to determine the last page for the footer
\usepackage{extramarks} % Required for headers and footers
\usepackage{graphicx} % Required to insert images
\usepackage{lipsum} % Used for inserting dummy 'Lorem ipsum' text into the template
\usepackage{amsmath, amsfonts, bm, physics}
\usepackage{xcolor}
\usepackage{listings}
\usepackage{hyperref}
\usepackage[toc,page]{appendix}

\lstset{
    %numbers=left,
    stepnumber=1,    
    firstnumber=1,
    numberfirstline=true,
    basicstyle=\ttfamily,
    keywordstyle=\color{blue}\ttfamily,
    stringstyle=\color{red}\ttfamily,
    commentstyle=\color{green}\ttfamily,
    breaklines=true,
}

% Margins
\topmargin=-0.45in
\evensidemargin=0in
\oddsidemargin=0in
\textwidth=6.5in
\textheight=9.0in
\headsep=0.25in 

\linespread{1.1} % Line spacing

% Set up the header and footer
\pagestyle{fancy}
\lhead{\hmwkAuthorName} % Top left header
\chead{\hmwkClass\ (\hmwkClassInstructor\ \hmwkClassTime): \hmwkTitle} % Top center header
\rhead{\firstxmark} % Top right header
\lfoot{\lastxmark} % Bottom left footer
\cfoot{} % Bottom center footer
\rfoot{Page\ \thepage\ of\ \pageref{LastPage}} % Bottom right footer
\renewcommand\headrulewidth{0.4pt} % Size of the header rule
\renewcommand\footrulewidth{0.4pt} % Size of the footer rule

\setlength\parindent{0pt} % Removes all indentation from paragraphs

%----------------------------------------------------------------------------------------
% DOCUMENT STRUCTURE COMMANDS
% Skip this unless you know what you're doing
%----------------------------------------------------------------------------------------

% Header and footer for when a page split occurs within a problem environment
\newcommand{\enterProblemHeader}[1]{
  \nobreak\extramarks{#1}{#1 continued on next page\ldots}\nobreak
  \nobreak\extramarks{#1 (continued)}{#1 continued on next page\ldots}\nobreak
}

% Header and footer for when a page split occurs between problem environments
\newcommand{\exitProblemHeader}[1]{
  \nobreak\extramarks{#1 (continued)}{#1 continued on next page\ldots}\nobreak
  \nobreak\extramarks{#1}{}\nobreak
}

\setcounter{secnumdepth}{0} % Removes default section numbers
\newcounter{homeworkProblemCounter} % Creates a counter to keep track of the number of problems

\newcommand{\homeworkProblemName}{}
\newenvironment{homeworkProblem}[1][Problem \arabic{homeworkProblemCounter}]{ % Makes a new environment called homeworkProblem which takes 1 argument (custom name) but the default is "Problem #"
    \stepcounter{homeworkProblemCounter} % Increase counter for number of
% problems
    \renewcommand{\homeworkProblemName}{#1} % Assign \homeworkProblemName the
% name of the problem
    \section{\homeworkProblemName} % Make a section in the document with the
% custom problem count
    \enterProblemHeader{\homeworkProblemName} % Header and footer within the
% environment
}{
    \exitProblemHeader{\homeworkProblemName} % Header and footer after the
% environment
}

\newcommand{\problemAnswer}[1]{ % Defines the problem answer command with the content as the only argument
    \noindent\textbf{\emph{Answer: }}#1 % Just put a keyword Answer in
    % bold/italic at the beginning
}

\newcommand{\homeworkSectionName}{}
\newenvironment{homeworkSection}[1]{ % New environment for sections within homework problems, takes 1 argument - the name of the section
    \renewcommand{\homeworkSectionName}{#1} % Assign \homeworkSectionName to the
% name of the section from the environment argument
    \subsection{\homeworkSectionName} % Make a subsection with the custom name
% of the subsection
    \enterProblemHeader{\homeworkProblemName\ [\homeworkSectionName]} % Header
% and footer within the environment
}{
    \enterProblemHeader{\homeworkProblemName} % Header and footer after the
% environment
}

\newtheorem{theorem}{Theorem}[homeworkProblemCounter]
\newtheorem{lemma}[theorem]{Lemma}
\newtheorem{proposition}[theorem]{Proposition}
\newtheorem{corollary}[theorem]{Corollary}

\newenvironment{proof}[1][Proof]{
  \begin{trivlist}
    \item[\hskip \labelsep {\bfseries #1}]}{
  \end{trivlist}
}
\newenvironment{definition}[1][Definition]{
  \begin{trivlist}
    \item[\hskip \labelsep {\bfseries #1}]}{
  \end{trivlist}
}

\newenvironment{example}[1][Example]{
  \begin{trivlist}
    \item[\hskip \labelsep {\bfseries #1}]}{
  \end{trivlist}
}
    
\newenvironment{remark}[1][Remark]{
  \begin{trivlist}
    \item[\hskip \labelsep {\bfseries #1}]}{
  \end{trivlist}
}

\newcommand{\qed}{
  \nobreak \ifvmode \relax \else
  \ifdim\lastskip<1.5em \hskip-\lastskip
  \hskip1.5em plus0em minus0.5em \fi \nobreak
  \vrule height0.75em width0.5em depth0.25em\fi
}

\lstset{
  frame=single,
  breaklines=true,
  postbreak=\raisebox{0ex}[0ex][0ex]{\ensuremath{\color{red}\hookrightarrow\space}}
}
   
%----------------------------------------------------------------------------------------
% NAME AND CLASS SECTION
%----------------------------------------------------------------------------------------

\newcommand{\hmwkTitle}{Assignment\ \#1} % Assignment title
\newcommand{\hmwkDueDate}{Monday, April\ 18,\ 2016} % Due date
\newcommand{\hmwkClass}{MAT\ 280} % Course/class
\newcommand{\hmwkClassTime}{MF 13:30 - 15:00} % Class/lecture time
\newcommand{\hmwkClassInstructor}{Prof. Thomas Strohmer} % Teacher/lecturer
\newcommand{\hmwkAuthorName}{Wenhao Wu} % Your name

%----------------------------------------------------------------------------------------
% TITLE PAGE
%----------------------------------------------------------------------------------------

\title{
  \vspace{2in}
  \textmd{\textbf{\hmwkClass:\ \hmwkTitle}}\\
  \normalsize\vspace{0.1in}\small{Due\ on\ \hmwkDueDate}\\
  \vspace{0.1in}\large{\textit{\hmwkClassInstructor\ \hmwkClassTime}}
  \vspace{3in}
}

\author{\textbf{\hmwkAuthorName}}
\date{} % Insert date here if you want it to appear below your name

%----------------------------------------------------------------------------------------

\begin{document}

  \maketitle
  
  %----------------------------------------------------------------------------------------
  % TABLE OF CONTENTS
  %----------------------------------------------------------------------------------------
  
  %\setcounter{tocdepth}{1} % Uncomment this line if you don't want subsections listed in the ToC
  
  \newpage
  \tableofcontents
  \newpage
  
  %----------------------------------------------------------------------------------------
  % PROBLEM 1
  %----------------------------------------------------------------------------------------
  \begin{homeworkProblem}
    Show that two random vectors in high dimensions are almost orthogonal.
    \vspace{10pt}
      
    \problemAnswer{
      Assuming to random variables $\mathbf{x}, \mathbf{y}\in\mathbb{R}^d$ are
      independently distributed. Also we assume that the $d$ entries of $X$
      are all i.i.d with 0 mean and non-zero variance of $\sigma_x^2$, and the
      $d$ entries of $Y$ are also i.i.d with 0 mean and non-zero variance of
      $\sigma_y^2$.
      Then the correlation coefficients between $\mathbf{x}$ and $\mathbf{y}$ is
      \begin{align}
        \rho & = \frac{\mathbf{x}^T\mathbf{y}}
        {\|\mathbf{x}\|_2^{1/2}\|\mathbf{y}\|_2^{1/2}} =
        \frac{\frac{1}{d}\sum_{i=1}^d x_iy_i} {\left(\frac{1}{d}\sum_{i=1}^d
        x_i^2\right)^{1/2} \left(\frac{1}{d}\sum_{i=1}^d y_i^2\right)^{1/2}}
      \end{align}
      
      According to the strong law of large number, as $d\rightarrow \infty$ we
      have
      \begin{align}
        \frac{1}{d}\sum_{i=1}^d x_iy_i \overset{a.s.}{\to} 0,\;
        \frac{1}{d}\sum_{i=1}^d x_i^2 \overset{a.s.}{\to} \sigma_x^2,\;
        \frac{1}{d}\sum_{i=1}^d y_i^2 \overset{a.s.}{\to} \sigma_y^2,\;
      \end{align}
      According to the continuous mapping theorem, we have 
      \begin{align}
        \rho \overset{a.s.}{\to} \frac{0}{\sqrt{\sigma_x^2\sigma_y^2}} = 0
      \end{align}
      which means $\mathbf{x}$ and $\mathbf{y}$ are orthogonal almost surely as
      $d\rightarrow \infty$.
    }
  \end{homeworkProblem}
  %\clearpage
  
  %----------------------------------------------------------------------------------------
  % PROBLEM 2
  %----------------------------------------------------------------------------------------
  \begin{homeworkProblem}
    Consider the following setup. Given a square of side length 1, we place four
    circles in the square as depicted in Figure 1 (each of the gray circles has
    radius 1/4). We now place a circle at the center of the square (the blue
    circle in Figure~\ref{fig:circles}(a)) such that this circle in the middle
    touches each of the four identical circles. Let $r$ denote the radius of the blue
    circle.
    
    We can do something analogous in three dimensions, see
    Figure~\ref{fig:circles}(b). We place eight spheres of radius 1/4 inside a
    cube of side length 1, and put a (blue) sphere in the middle such that it
    touches all eight (gray) spheres.
    
    In four dimensions we can arrange 16 hyperspheres of radius 1/4 inside a
    hypercube of side length 1 and place a hypersphere in the middle, so that
    this hypersphere all the other 16 hyperspheres.

    Obviously we can do this for increasing dimension $d$. What happens with the
    blue hypersphere in the middle as d increases? Will it shrink? Will it be of
    constant size? Will it grow outside the hypercube?
    
    (Hint: Check the diameter of the blue hypersphere in comparison to the
    sidelength of the cube as d increases. This is actually not difficult to
    compute, it may sound more complicated than it is).
    \begin{figure}[htb]
      \centering
      \begin{minipage}[b]{0.45\columnwidth}
        \centering
        \centerline{\includegraphics[width=3cm]{./figs/fig1.pdf}}
        \centerline{(a) 4 circles}\medskip
      \end{minipage}
      \hfill
      \begin{minipage}[b]{0.45\columnwidth}
        \centering
        \centerline{\includegraphics[width=3cm]{./figs/fig2.pdf}}
        \centerline{(b) 8 circles}\medskip
      \end{minipage}
      \caption{Circles in a cube.}
      \label{fig:circles}
    \end{figure}
    
    \vspace{10pt}
      
    \problemAnswer{
      In a unit cube of dimension $d$, the centers of the corner balls are in
      $[\pm 1/4, \ldots, \pm 1/4]^T$, of which the distances w.r.t the origin is
      $d_c = \sqrt{d}/4$. Consequently, the radius of the center ball which
      touches all the corner balls is $r = d_c - 1/4 = \sqrt{d}/4 - 1/4$, which
      apprarently increases as $d$ increases. Since the unit cube has a fixed
      edge length of 1, the centerball will grow outside the cube.
    }
    
  \end{homeworkProblem}
  
  %----------------------------------------------------------------------------------------
  % PROBLEM 3
  %----------------------------------------------------------------------------------------
  \begin{homeworkProblem}
    Show that for every fixed dimension reduction matrix $\mathbf{A}$ of size
    $k\times d$ with $k < d$, there exists vectors $\mathbf{x}, \mathbf{y} \in
    \mathbb{R}^d$ such that the distance $\|\mathbf{Ax} - \mathbf{Ay}\|$ (no
    matter which norm we use) is vastly different from $\|\mathbf{x} -
    \mathbf{y}\|$.
    
    \vspace{10pt}
      
    \problemAnswer{
      Since $k < d$, there always exists $(\mathbf{x} -
      \mathbf{y})\in \mathrm{null}(\mathbf{A})$ and $(\mathbf{x} -
      \mathbf{y})\not=\mathbf{0}$. Consequently, $\forall N>0$ and
      under arbitrary norm, there exists $\mathbf{x}, \mathbf{y}$ such that
      \begin{align}
      \|\mathbf{x} - \mathbf{y}\| > N,\; \|\mathbf{Ax} - \mathbf{Ay}\| = 0.
      \end{align}
      thus they are vastly different.
    }
  \end{homeworkProblem}
  
  
  %----------------------------------------------------------------------------------------
  % PROBLEM 4
  %----------------------------------------------------------------------------------------
  \begin{homeworkProblem}
    The Yale Face Database contains images from various individuals in different
    poses and under different lighting conditions. Some of the images are stored
    in the file \texttt{SomeYaleFaces.mat}.
    
    Load this file into Matlab. The variable $\mathbf{X}$ is a matrix of size
    $1024 \times 2414$. Each column of $\mathbf{X}$ is an image of size $32
    \times 32$ (in vectorized form). The 2414 columns are images of 38 different
    persons in about 64 poses each. You can easily covert the $k$-th column of
    $\mathbf{X}$ back to an image via the commands
\begin{lstlisting}[language=Matlab]
xk = X(:,k); xk = reshape(xk,32,32);
\end{lstlisting} 
    The command
\begin{lstlisting}[language=Matlab]
imagesc(x1); colormap(gray);
\end{lstlisting}
    will display the image.
    
    You can conveniently display multiple images if you want with the file
    \texttt{showfaces.m}.
    
    We want to compare three dimension reduction methods by comparing how well
    distances between the differen images are preserved: (i)
    Johnson-Lindenstrauss projection, (ii) Fast Johnson Lindenstrauss projection
    and (iii) simple random sampling (i.e., randomly picking $k$ indices).
    
    Choose different values for the reduced dimension $k$ and compare the
    dimension reduction ability of the three methods. You need to think about
    how to devise such an experiment. There are of course multiple options to do
    so.
    
    \vspace{10pt}
      
    \problemAnswer{
      To compare the accuracy of the three dimension reduction methods, we adopt
      Sammon's stress measure~\cite[Eq.(1)]{sammon1969nonlinear}. For each
      dimension reduction scheme, $N=20$ sets of random matrices/selections are
      generated and applied to each of the 2414 images and the average error is
      plotted in Figure~\ref{fig:dimension_reduction}. As we can see, for small and
      moderate value of $k$, Fast Johnson-Lindenstrauss (FJL) significantly
      outperform the other 2 dimension reduction schemes in terms of accuracy.
      As $k\rightarrow 1024$, we notice that the Johnson-Lindenstrauss (JL)
      has a lower error rate than the other two methods. This is because JL
      strictly enforces orthogonal projection, while for FJL and random
      selection (RS) the random seletion of $k$ rows does not guarantee
      orthogonality.
      
      \begin{figure}[htb]
        \centering
        \includegraphics[width=0.75\columnwidth]{figs/dimension_reduction.pdf}
        \caption{Comparison between Johnson-Lindenstrauss (JL) projection,
        Fast Johnson-Lindenstrauss (FJL) projection with fast walsh hadamard
        transform (FWHT) and random selection (RS).}
        \label{fig:dimension_reduction}
    \end{figure}
        
      The matlab code for this simulation is as follows:
      \lstinputlisting[language=Matlab]{hw1_4.m}
      \lstinputlisting[language=Matlab]{get_dist.m}
      \lstinputlisting[language=Matlab]{get_error.m}
    }
    
  \end{homeworkProblem}
  %\newpage
  %\begin{appendices} 
  %\end{appendices}
  \bibliographystyle{unsrt}
  \bibliography{refs}
  
  %----------------------------------------------------------------------------------------

\end{document}