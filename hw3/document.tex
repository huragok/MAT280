%%%%%%%%%%%%%%%%%%%%%%%%%%%%%%%%%%%%%%%%%
% Structured General Purpose Assignment
% LaTeX Template
%
% This template has been downloaded from:
% http://www.latextemplates.com
%
% Original author:
% Ted Pavlic (http://www.tedpavlic.com)
%
% Note:
% The \lipsum[#] commands throughout this template generate dummy text
% to fill the template out. These commands should all be removed when 
% writing assignment content.
%
%%%%%%%%%%%%%%%%%%%%%%%%%%%%%%%%%%%%%%%%%

%----------------------------------------------------------------------------------------
% PACKAGES AND OTHER DOCUMENT CONFIGURATIONS
%----------------------------------------------------------------------------------------

\documentclass{article}

\usepackage{fancyhdr} % Required for custom headers
\usepackage{lastpage} % Required to determine the last page for the footer
\usepackage{extramarks} % Required for headers and footers
\usepackage{graphicx} % Required to insert images
\usepackage{lipsum} % Used for inserting dummy 'Lorem ipsum' text into the template
\usepackage{amsmath, amsfonts, bm, physics}
\usepackage{xcolor}
\usepackage{listings}
\usepackage{hyperref}
\usepackage[toc,page]{appendix}
\usepackage{steinmetz}

\lstset{
    %numbers=left,
    stepnumber=1,    
    firstnumber=1,
    numberfirstline=true,
    basicstyle=\ttfamily,
    keywordstyle=\color{blue}\ttfamily,
    stringstyle=\color{red}\ttfamily,
    commentstyle=\color{green}\ttfamily,
    breaklines=true,
}

% Margins
\topmargin=-0.45in
\evensidemargin=0in
\oddsidemargin=0in
\textwidth=6.5in
\textheight=9.0in
\headsep=0.25in 

\linespread{1.1} % Line spacing

% Set up the header and footer
\pagestyle{fancy}
\lhead{\hmwkAuthorName} % Top left header
\chead{\hmwkClass\: \hmwkTitle} % Top center header
\rhead{\firstxmark} % Top right header
\lfoot{\lastxmark} % Bottom left footer
\cfoot{} % Bottom center footer
\rfoot{Page\ \thepage\ of\ \pageref{LastPage}} % Bottom right footer
\renewcommand\headrulewidth{0.4pt} % Size of the header rule
\renewcommand\footrulewidth{0.4pt} % Size of the footer rule

\setlength\parindent{0pt} % Removes all indentation from paragraphs

%----------------------------------------------------------------------------------------
% DOCUMENT STRUCTURE COMMANDS
% Skip this unless you know what you're doing
%----------------------------------------------------------------------------------------

% Header and footer for when a page split occurs within a problem environment
\newcommand{\enterProblemHeader}[1]{
  \nobreak\extramarks{#1}{#1 continued on next page\ldots}\nobreak
  \nobreak\extramarks{#1 (continued)}{#1 continued on next page\ldots}\nobreak
}

% Header and footer for when a page split occurs between problem environments
\newcommand{\exitProblemHeader}[1]{
  \nobreak\extramarks{#1 (continued)}{#1 continued on next page\ldots}\nobreak
  \nobreak\extramarks{#1}{}\nobreak
}

\setcounter{secnumdepth}{0} % Removes default section numbers
\newcounter{homeworkProblemCounter} % Creates a counter to keep track of the number of problems

\newcommand{\homeworkProblemName}{}
\newenvironment{homeworkProblem}[1][Problem \arabic{homeworkProblemCounter}]{ % Makes a new environment called homeworkProblem which takes 1 argument (custom name) but the default is "Problem #"
    \stepcounter{homeworkProblemCounter} % Increase counter for number of
% problems
    \renewcommand{\homeworkProblemName}{#1} % Assign \homeworkProblemName the
% name of the problem
    \section{\homeworkProblemName} % Make a section in the document with the
% custom problem count
    \enterProblemHeader{\homeworkProblemName} % Header and footer within the
% environment
}{
    \exitProblemHeader{\homeworkProblemName} % Header and footer after the
% environment
}

\newcommand{\problemAnswer}[1]{ % Defines the problem answer command with the content as the only argument
    \noindent\textbf{\emph{Answer: }}#1 % Just put a keyword Answer in
    % bold/italic at the beginning
}

\newcommand{\homeworkSectionName}{}
\newenvironment{homeworkSection}[1]{ % New environment for sections within homework problems, takes 1 argument - the name of the section
    \renewcommand{\homeworkSectionName}{#1} % Assign \homeworkSectionName to the
% name of the section from the environment argument
    \subsection{\homeworkSectionName} % Make a subsection with the custom name
% of the subsection
    \enterProblemHeader{\homeworkProblemName\ [\homeworkSectionName]} % Header
% and footer within the environment
}{
    \enterProblemHeader{\homeworkProblemName} % Header and footer after the
% environment
}

\newtheorem{theorem}{Theorem}[homeworkProblemCounter]
\newtheorem{lemma}[theorem]{Lemma}
\newtheorem{proposition}[theorem]{Proposition}
\newtheorem{corollary}[theorem]{Corollary}

\newenvironment{proof}[1][Proof]{
  \begin{trivlist}
    \item[\hskip \labelsep {\bfseries #1}]}{
  \end{trivlist}
}
\newenvironment{definition}[1][Definition]{
  \begin{trivlist}
    \item[\hskip \labelsep {\bfseries #1}]}{
  \end{trivlist}
}

\newenvironment{example}[1][Example]{
  \begin{trivlist}
    \item[\hskip \labelsep {\bfseries #1}]}{
  \end{trivlist}
}
    
\newenvironment{remark}[1][Remark]{
  \begin{trivlist}
    \item[\hskip \labelsep {\bfseries #1}]}{
  \end{trivlist}
}

\newcommand{\qed}{
  \nobreak \ifvmode \relax \else
  \ifdim\lastskip<1.5em \hskip-\lastskip
  \hskip1.5em plus0em minus0.5em \fi \nobreak
  \vrule height0.75em width0.5em depth0.25em\fi
}

\lstset{
  frame=single,
  breaklines=true,
  postbreak=\raisebox{0ex}[0ex][0ex]{\ensuremath{\color{red}\hookrightarrow\space}}
}
   
%----------------------------------------------------------------------------------------
% NAME AND CLASS SECTION
%----------------------------------------------------------------------------------------

\newcommand{\hmwkTitle}{Assignment\ \#3} % Assignment title
\newcommand{\hmwkDueDate}{Monday, May\ 16,\ 2016} % Due date
\newcommand{\hmwkClass}{MAT\ 280} % Course/class
\newcommand{\hmwkClassTime}{MF 13:30 - 15:00} % Class/lecture time
\newcommand{\hmwkClassInstructor}{Prof. Thomas Strohmer} % Teacher/lecturer
\newcommand{\hmwkAuthorName}{Wenhao Wu} % Your name

%----------------------------------------------------------------------------------------
% TITLE PAGE
%----------------------------------------------------------------------------------------

\title{
  \vspace{2in}
  \textmd{\textbf{\hmwkClass:\ \hmwkTitle}}\\
  \normalsize\vspace{0.1in}\small{Due\ on\ \hmwkDueDate}\\
  \vspace{0.1in}\large{\textit{\hmwkClassInstructor\ \hmwkClassTime}}
  \vspace{3in}
}

\author{\textbf{\hmwkAuthorName}}
\date{} % Insert date here if you want it to appear below your name

%----------------------------------------------------------------------------------------

\begin{document}

  \maketitle
  
  %----------------------------------------------------------------------------------------
  % TABLE OF CONTENTS
  %----------------------------------------------------------------------------------------
  
  %\setcounter{tocdepth}{1} % Uncomment this line if you don't want subsections listed in the ToC
  
  \newpage
  \tableofcontents
  \newpage
  
  %----------------------------------------------------------------------------------------
  % PROBLEM 1
  %----------------------------------------------------------------------------------------
  \begin{homeworkProblem}
    Download the dataset \texttt{crescents.mat} from the Class website and load
    it into Matlab. It contains two-hundred points in two dimensions. If you
    plot it, you see that the data form two half-moon-like clusters. Clearly,
    k-means directly applied to this dataset will fail to cluster the data
    according to these two shapes. Use the graph Laplacian or diffusion maps
    (followed by k-means) to try to cluster the data as good as possible
    according to the half-moon shapes. You can use Matlab’s k-means function to
    do the actual clustering once you transformed the data.
    \vspace{10pt}
      
    \problemAnswer{
      
    }
  \end{homeworkProblem}
  %\clearpage
  
  %----------------------------------------------------------------------------------------
  % PROBLEM 2
  %----------------------------------------------------------------------------------------
  \begin{homeworkProblem}
    Download the dataset \texttt{genomedata.mat} from the Class website; it
    contains Single Nucleid Polymorphisms data from the Human Genome Diversity
    Project. The data consists of an array consisting of 5000 rows, each row has
    1043 different strings. The 5000 rows are Single Nucleid Polymorphisms, the
    columns correspond to 1043 different individuals. The entries are not
    numerical values (quite annoyingly), but contain the characters
    `AA', `CC', `GG', `TT', `AG', `AC', `TC', `TG', and (even more annoyingly)
    also `--', the latter represents missing measurements. Your goal is to
    cluster the data into a small set of clusters. After loading the file into
    Matlab, you need to convert the characters into numerical values. It is up
    to you which conversion you use (you can use the file \texttt{gen2vec.m} to
    do the actual conversion, once you have chosen a conversion rule).

    Since the data are high-dimensional you first need to reduce the dimension
    before clustering. You should attempt the dimension reduction via PCA as
    well as via diffusion maps. In both cases you need to decide how many
    dimensions you want to use. Also, in both cases you may want to use Matlab’s
    k-means function to do the actual clustering after dimension reduction.
    
    Note: Your results may differ from mine, because you will likely choose a
    different conversion rule. I did not get a meaningful clustering via PCA,
    but did achieve reasonable clustering via diffusion maps.
    \vspace{10pt}
      
    \problemAnswer{
      
    }
  \end{homeworkProblem}
  
  %\newpage
  %\begin{appendices} 
  %\end{appendices}
  %$\bibliographystyle{unsrt}
  %\bibliography{refs}
  
  %----------------------------------------------------------------------------------------

\end{document}